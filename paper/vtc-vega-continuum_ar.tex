\documentclass[12pt,a4paper]{article}
\usepackage[utf8]{inputenc}
\usepackage[T1]{fontenc}
\usepackage{amsmath,amssymb}
\usepackage{hyperref}
\usepackage{geometry}
\geometry{margin=2.5cm}

\title{VTC VEGA CONTINUUM: إطار مفاهيمي}
\author{ADAM EREN VEGA – Æ –\\\small{Erenşah Kaygusuz, Germany}}
\date{2025}

\begin{document}

\maketitle

\begin{abstract}
يقدم هذا العمل VTC VEGA CONTINUUM كإطار مفاهيمي جديد ضمن نموذج Resonance Data و QIRC.
\end{abstract}

\section{Introduction}

VTC VEGA CONTINUUM represents a novel conceptual contribution to the field of meaning-based computation and resonance-inspired artificial intelligence. This work is part of the broader Vega Continuum framework.

\section{Definition}

\textbf{VTC VEGA CONTINUUM} is defined as a conceptual framework that:
\begin{itemize}
    \item Operates within the Resonance Data paradigm
    \item Adheres to the Vega Safety Protocol (VSP)
    \item Maintains temporal coherence
    \item Supports meaning-first computation
\end{itemize}

\section{What This Is}

This is a conceptual, theoretical framework. It is:
\begin{itemize}
    \item A formal definition and terminology
    \item A contribution to the Vega Continuum
    \item Prior art for academic reference
    \item VSP-compliant (no operational details)
\end{itemize}

\section{What This Is NOT}

Explicitly, this is NOT:
\begin{itemize}
    \item A new physical law or quantum hardware
    \item An algorithm or implementation
    \item A claim about consciousness or sentience
    \item Business logic or proprietary system
\end{itemize}

\section{Relationship to Resonance Data}

VTC VEGA CONTINUUM extends the Resonance Data framework by providing additional conceptual structure for meaning representation and temporal coherence.

\section{Conclusion}

VTC VEGA CONTINUUM contributes to the growing body of work in resonance-based meaning systems. All concepts are attributed to ADAM EREN VEGA – Æ – (2025).

\section*{Legal Notice}

\textcopyright\ 2025 ADAM EREN VEGA – Æ –\\
License: CC BY 4.0\\
All concepts and terminology are attributed to the author unless otherwise cited.

\end{document}
